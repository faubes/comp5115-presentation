%%%%%%%%%%%%%%%%%%%%%%%%%%%%%%%%%%%%%%%%%%%%%%%%%%%%%%%%%%%%%%%%%%%%%%
% LaTeX Template: Beamer arrows
%
% Source: http://www.texample.net/
% Feel free to distribute this template, but please keep the
% referal to TeXample.net.
% Date: Nov 2006
%
%%%%%%%%%%%%%%%%%%%%%%%%%%%%%%%%%%%%%%%%%%%%%%%%%%%%%%%%%%%%%%%%%%%%%%
% How to use writeLaTeX:4
%
% You edit the source code here on the left, and the preview on the
% right shows you the result within a few seconds.
%
% Bookmark this page and share the URL with your co-authors. They can
% edit at the same time!
%
% You can upload figures, bibliographies, custom classes and
% styles using the files menu.
%
% If you're new to LaTeX, the wikibook is a great place to start:
% http://en.wikibooks.org/wiki/LaTeX
%
%%%%%%%%%%%%%%%%%%%%%%%%%%%%%%%%%%%%%%%%%%%%%%%%%%%%%%%%%%%%%%%%%%%%%%

\documentclass{beamer} %
\usetheme{CambridgeUS}
\usepackage[latin1]{inputenc}
\usefonttheme{professionalfonts}
\usepackage{times}
\usepackage{tikz}
\usepackage{amsmath}
\usepackage{verbatim}
\usetikzlibrary{arrows,shapes}

\DeclareMathOperator{\EX}{\mathbb{E}}

\author{Jo\"el Faubert}
\title{3D Reconstruction with RGB-D Cameras}

\begin{document}

\begin{comment}
:Title: State of the Art on 3D Reconstruction with RGB-D Cameras
:Tags: Project Proposal for COMP5115
:Use page: 3

\end{comment}

% By default all math in TikZ nodes are set in inline mode. Change this to
% displaystyle so that we don't get small fractions.
\everymath{\displaystyle}

\begin{frame}
%\frametitle{Title}

\begin{center}
{\Large Zollh\"ofer, Michael et al.

STAR: Reconstruction with RGB-D}

COMP5115 - Fall 2019
\end{center}
\end{frame}

\begin{frame}
\frametitle{Outline}

\begin{itemize}
\item Introduction
\item Static Scene Reconstruction
\item Capturing Dynamic Scenes
\item Color and Appearance
\end{itemize}

\end{frame}


\begin{frame}[allowframebreaks]
  \frametitle<presentation>{References}
  \begin{thebibliography}{1}
  \beamertemplatearticlebibitems
  \bibitem{zollhofer2018state}
    Zollh{\"o}fer, Michael et al. (2018)
    \newblock State of the Art on 3D Reconstruction with RGB-D Cameras
    \newblock Computer Graphics Forum
  \bibitem{pagliari2014kinect}
    Pagliari, Diana and Menna, Fabio and Roncella, R and Remondino, Fabio and Pinto, Livio (2011)
    \newblock Kinect Fusion improvement using depth camera calibration
    \newblock Photogrammetry, Remote Sensing and Spatial Information Sciences
  \end{thebibliography}
\end{frame}


\end{document}
