\begin{frame}
%\frametitle{Title}

\begin{center}
{\Large
Pagliari et al.

Kinect Fusion Improvement Using Depth Camera Calibration}

COMP5115 - Fall 2019
\end{center}
\end{frame}

\begin{frame}
\frametitle{Outline}

\begin{itemize}
\item Introduction and Motivation
\item Related Works
\item The KinectFusion Method
\item Results and Quick Look at State of the Art
\end{itemize}

\end{frame}

\begin{frame}
\frametitle{Introduction and Motivation}

\begin{itemize}
\item Purchased an Intel RealSense D435 Camera.
\item Studied STAR paper by Zollh\"ofer to see how it could be used.
\item Realized that article was too high-level (and advanced).
\item KinectFusion seems to have established current paradigm. Explains math bits nicely,
so a good starting paper.
\end{itemize}

[depth image here]

\end{frame}

\begin{frame}
\frametitle{Problem Statement}

Problem: process a stream of RGB-D frames for Simultaneous Localization and Mapping
(and do it in real time!)

\begin{itemize}
  \item Tracking: estimate the pose (position + orientation) of the camera.
  Camera presumed moving through space -- need to keep track of position and which way it's pointing.
  \item Mapping: (incrementally) build a model of the scene captured by camera.
\end{itemize}
\end{frame}

\begin{frame}
\frametitle{Challenges}
\begin{itemize}
  \item High volume of data (640x480 @ 30fps = 9 million points per sec)
  \item Occlusion (stuff in the way), holes
  \item Measurement errors: incident angles, shiny or transparent materials
  \item Potentially erratic camera movement: blurry measurements
  \item Dynamic scenes, moving objects
  \item Camera drift: accumulation of errors in pose estimation
\end{itemize}
\end{frame}

\begin{frame}[allowframebreaks]
  \frametitle<presentation>{Related Works}
  \begin{thebibliography}{1}
  \beamertemplatearticlebibitems
  \bibitem{murartal2015slam}
    \newblock Mur-Artal R., Montiel J. M. M., Tardos J. D.
    \newblock Orb-slam: a versatile and accurate monocular SLAM system 2015.
  \bibitem{bogo2015moving}
    \newblock Bogo F., Black M. J., Loper M., Romero J
    \newblock Detailed full-body reconstructions of moving people from monocular RGB-D sequences.
  \end{thebibliography}
\end{frame}
